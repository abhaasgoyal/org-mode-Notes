% Created 2020-06-21 Sun 20:06
% Intended LaTeX compiler: xelatex
\documentclass[11pt]{article}
\usepackage{graphicx}
\usepackage{grffile}
\usepackage{longtable}
\usepackage{wrapfig}
\usepackage{rotating}
\usepackage[normalem]{ulem}
\usepackage{amsmath}
\usepackage{textcomp}
\usepackage{amssymb}
\usepackage{capt-of}
\usepackage{hyperref}
\author{Abhaas Goyal}
\date{\today}
\title{Git from the Ground Up}
\hypersetup{
 pdfauthor={Abhaas Goyal},
 pdftitle={Git from the Ground Up},
 pdfkeywords={},
 pdfsubject={},
 pdfcreator={Emacs 26.3 (Org mode 9.1.9)}, 
 pdflang={English}}
\begin{document}

\maketitle
\tableofcontents

\section{Introduction}
\label{sec:org0fe7799}
We're going to learn a little something about the git data model by crafting the \texttt{.git} directory and  blob objects by hand. It'll be fun! Also exporting beautiful org files into \href{https://github.com/fniessen/org-html-themes}{Awesome HTML} ( Cloned and used in \texttt{\#+SETUPFILE})

\section{Heading 1}
\label{sec:org9dc6a58}
bold
\section{Heading 2}
\label{sec:org807bc0d}
\emph{italics}
\section{Heading 3}
\label{sec:org144f4b3}
\texttt{monospace}

\section{Where are we, and is git happy?}
\label{sec:orgf78f9d8}
Let's start by seeing where we are, and if we're in a valid git repository.

\begin{verbatim}
echo "My current working dir is: $(dirs +0)"
ls -l
\end{verbatim}
\end{document}
